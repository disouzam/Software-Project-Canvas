% Comandos criados

% star - cannot contain \par (see https://tex.stackexchange.com/questions/1050/whats-the-difference-between-newcommand-and-newcommand)
\newcommand*{\gxdefine}[1]{\textbf{#1}\index{#1|textbf}}
\newcommand*{\gxdefinei}[1]{\textit{\textbf{#1}}\index{#1|textbf}}
\newcommand*{\gxindex}[1]{#1\index{#1}}
\newcommand*{\gxindexi}[1]{\textit{#1}\index{#1}}
\newcommand*{\gxdefineplus}[2][]{\textbf{#2 #1}\index{#2@\textit{#2}\ifx#1\empty\else!\textit{#1}\fi}}



\newcommand{\Excel}{Excel\texttrademark}
\newcommand{\ExcelScale}{0.4}


% can´t remember what it does
\makeatletter
\DeclareRobustCommand*{\escapeus}[1]{%
    \begingroup\@activeus\scantokens{#1\endinput}\endgroup}
\begingroup\lccode`\~=`\_\relax
    \lowercase{\endgroup\def\@activeus{\catcode`\_=\active \let~\_}}
\makeatother

%% DESENHA CUBOS!
\newcommand{\drawbox}[5]{
    \pgfmathsetmacro \angle {30}
    \pgfmathsetmacro \xd {{2/3*cos(\angle)*#5}}
    \pgfmathsetmacro \yd {{2/3*sin(\angle)*#5}}
    \pgfmathsetmacro \x {{#1-#5+(#2-#5)*(\xd)*#5}}
    \pgfmathsetmacro \y {{#3-#5+(#2-#5)*(\yd)*#5}}

    \draw[fill=#4] (\x,\y) -- (\x+#5,\y) -- (\x+#5,\y+#5) -- (\x,\y+#5) -- cycle;

    \draw[fill=#4] (\x,\y+#5) -- (\x+\xd,\y+#5+\yd) -- (\x+#5+\xd,\y+#5+\yd) -- (\x+#5,\y+#5) -- cycle;

    \draw[fill=#4] (\x+#5,\y+#5) -- (\x+#5+\xd,\y+#5+\yd) -- (\x+#5+\xd,\y+\yd) -- (\x+#5,\y) -- cycle;
}

% desenha linhas malucas
\newcommand\irregularline[2]{%

%  let \n1 = {rand*(#1)} in  +(0,\n1)

  \foreach \a in {0.0,0.1,...,#2}{
    let \n1 = {rnd*(#1)} in
    -- +(\a,\n1)
  }
}  % #1=seed, #2=length of horizontal line

\setlength{\parskip}{.5em}

% scrbook permite mudar - o tipo de documento que estou usando
% e corrigir o erro
%Package tocbasic Warning: number width of figure toc entries should be
\DeclareTOCStyleEntry[numwidth=45pt]{tocline}{figure}
\DeclareTOCStyleEntry[numwidth=45pt]{tocline}{section}
\DeclareTOCStyleEntry[numwidth=45pt]{tocline}{subsection}


% comentários nas margens com alinhamento que faz sentido
%https://tex.stackexchange.com/questions/411939/marginpar-text-alignment/417276#417276
\newcommand{\alignedmarginpar}[1]{%
    \Ifthispageodd{%
        \marginpar{\raggedright\small #1}}{%
        \marginpar{\raggedleft\small #1}}%
    }


% Marcas usando bbding ou outro pacote
\newcommand{\gxok}{\textcolor{green}{\CheckmarkBold}}
\newcommand{\gxnot}{\textcolor{red}{\XSolidBold}}


% https://tex.stackexchange.com/questions/32683/rotated-column-titles-in-tabular
% cria um comando para rodar os headings das tabelas
% e acertar o tamanho das colunas,
% parametros rotacao (op) tamanho(opcional) e texto (mandatório)
% precisa do xparse
\NewDocumentCommand{\gxrot}{O{90} O{1em} m}{\makebox[#2][l]{\rotatebox{#1}{#3}}}%


\newcommand{\gxibfcolor}{NavyBlue}
\newcommand{\gxibbcolor}{White}


\newenvironment{gxwinfobox}[2][r]%
{%
\begin{wrapfigure}{#1}{0.5\textwidth}%
\begin{tcolorbox}[title=\textbf{\Info #2},colframe=\gxibfcolor,colback=\gxibbcolor]%
}%
{%
\end{tcolorbox}%
\end{wrapfigure}%
}%

%% Usando o newfloat
\DeclareFloatingEnvironment{floatbox}


\newenvironment{gxinfobox}[1]
{%
\begin{tcolorbox}[%
title={\Large\Info\ }\textbf{#1},%
colframe=\gxibfcolor,colback=\gxibbcolor, sharp corners=uphill]%
\parindent=.5cm \parskip=.5em%
}%
{%
\end{tcolorbox}%
}%


\newcommand{\gxcwinfobox}[3][r]{\begin{gxwinfobox}[#1]{#2}#3\end{gxwinfobox}}

\NewDocumentCommand{\whybox}{m+m}{
    \begin{tcolorbox}[title=\textbf{Por que #1?},colframe=blue!75!black,
        colback=White,fonttitle=\bfseries]
        \protect#2
\end{tcolorbox}}

\newif\ifLivroES%
\newif\ifLivroAS%
\newif\ifReportValor%
\newif\ifLivroADP%
\newif\ifLivroDW%
\newif\ifLivroRCBD
\LivroESfalse%
\LivroASfalse%
\LivroADPfalse%
\ReportValorfalse%
\LivroDWfalse%
\LivroRCBDfalse

\def\gxdoctopic{software}

\newcommand{\gxchoosedoctopic}[1]{%
    \def\gxdoctopic{#1\ } %
}%



% Widows e Orphans (estratégia 1)
\widowpenalty10000
\clubpenalty10000

\newcommand{\gxnewacronym}[2]{\item[#1] #2}%
\newcommand{\acronlist}%
{%
\chapter*{Lista de Acrônimos}%
\begin{description}[leftmargin=!,labelwidth=\widthof{\bfseries ABCDEFGH}]%
\input{Extras/acronimosrt}%
%}
%{
\end{description}%
}%

%% Trying Veelo Chapter Heading
%\newcommand{\mychapfontsize}{\fontsize{40}{55}\selectfont}
%\setkomafont{chapter}{\normalfont\mychapfontsize\raggedleft}
%\setkomafont{chapterprefix}{\normalfont\mychapfontsize}
%\let\raggedchapter\raggedleft
%\newlength{\htZahl}
%\renewcommand*{\chapterformat}{\MakeUppercase{\chapapp}\makebox[0pt][l]%
%{\hspace*{\marginparsep}\mychapfontsize\thechapter~\settoheight{\htZahl}%
%{\thechapter}\rule{15mm}{\htZahl}}}
%\RedeclareSectionCommand[beforeskip=5pt,afterskip=20pt]{chapter}
%% \renewcommand*{\chapterformat}{\MakeUppercase{\chapapp}\makebox[0pt][l]{\hspace*{\marginparsep}\Huge\thechapter~\rule{15mm}{\ht\strutbox}}}
%%\renewcommand*{\chapterformat}{%
%%  \makebox[\linewidth][r]{\MakeUppercase{\chapappifchapterprefix{}}}%
%%  \enskip\resizebox{!}{1.2cm}{\thechapter}\rlap{ \rule{15cm}{1.2cm} }%
%%}
%%
%%\newlength{\htZahl}
%%
%\RedeclareSectionCommand[beforeskip=5pt,afterskip=20pt]{chapter}


%\definecolor{chaptercolor}{RGB}{78,200,100}
%
%\renewcommand\raggedchapter{\raggedleft}
%\setkomafont{chapter}{\Huge}
%\setkomafont{chapterprefix}{\large}
%\newkomafont{chapternumber}{\mdseries\fontsize{50pt}{60pt}\selectfont}
%
%\tikzset{
%    headings/base/.style = {
%      outer sep = 0pt,
%      inner sep = 5pt,
%    },
%    headings/chapterbackground/.style = {
%      headings/base,
%      shade,
%      left color = white,
%      right color = chaptercolor,
%    },
%    headings/chapapp/.style = {
%      headings/base,
%      text = chaptercolor,
%      font = \usekomafont{chapterprefix}
%    },
%    headings/chapternumber/.style= {
%      headings/base,
%      text = chaptercolor,
%      font = \usekomafont{chapternumber}
%    },
%    headings/chapterline/.style = {
%      chaptercolor,
%      line width = 2pt
%    }
%}
%\makeatletter
%\renewcommand*\chapterlinesformat[3]{%
%  \ifstr{#1}{chapter}{%
%    \begin{tikzpicture}[baseline=(title.base)]
%      \node[headings/chapterbackground](title){%
%        \parbox[t]
%          {\dimexpr\textwidth-2\pgfkeysvalueof{/pgf/inner xsep}\relax}
%          {\raggedchapter #3}%
%      };
%      \node[headings/chapapp,anchor=south east]
%        at (title.north east){\ifstr{#2}{}{}{\chapapp}\strut};
%      \useasboundingbox
%        (current bounding box.north west)
%        rectangle
%        ([yshift=-10pt]current bounding box.south east);
%      \draw[headings/chapterline]
%        (current bounding box.south east)++(+.5\pgflinewidth,0)--+(0,\paperheight);
%      \node[anchor=base west,headings/chapternumber]
%        at([xshift=10pt]title.base-|current bounding box.east){#2};
%    \end{tikzpicture}
%    \par
%  }{%
%    \@hangfrom{#2}{#3}% other section levels using style=chapter
%  }%
%}
%\makeatother

\newcounter{exerciciocounter}
\setcounter{exerciciocounter}{0}
\counterwithin{exerciciocounter}{chapter}
\newenvironment{exercise}%
{%begdef
\refstepcounter{exerciciocounter}%
\par\medskip%
\textbf{Exercício~\theexerciciocounter:}%
\par%
}%
{%
\medskip%
}%
\newenvironment{solution}%
{%begdef
\textbf{Solução:}%
\medskip%
}%
{%
\medskip%
}%


\newcommand{\gxchapterd}[2][]{%
\DraftwatermarkOptions{stamp=true}%
\chapter{#2}%
\ifx#1\empty\else\label{#1}\fi
}%

\newcommand{\gxchapterf}[2][]{%
\DraftwatermarkOptions{stamp=false}%
\chapter{#2}%
\ifx#1\empty\else\label{#1}\fi
}%

\newcommand{\furl}[1]{\footnote{\url{#1}}}


% Criando um novo ambiente "myoutline"
\newenvironment{wbs}
{
    \begin{outline}[enumerate]
    \setlist[enumerate,1]{label=\arabic*.}
    \setlist[enumerate,2]{label=\arabic{enumi}.\arabic*.}
    \setlist[enumerate,3]{label=\arabic{enumi}.\arabic{enumii}.\arabic*.}
}
{
    \end{outline}
}

