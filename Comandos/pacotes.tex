% Pacotes usados
% from KOMA
\usepackage{scrhack} % evita warning do lst listings
\usepackage[brazilian]{babel}
\usepackage[T1]{fontenc}
% melhor typesetting
\usepackage{microtype}
%para underline que quebra linha
% usar \uline
% o normalem significa manter o \emph como é
% senão ele é alterado para underline
\usepackage[normalem]{ulem}

\usepackage[dvipsnames]{xcolor}
\usepackage{graphicx}

\usepackage{outlines}
% usa mais interfaces de saída
%  solution for the error “no room for a new \write” (This is a deep magic over TeX)
%\usepackage{morewrites}
%\morewritessetup{allocate=10}

% texto em português
% https://tex.stackexchange.com/questions/13172/detect-which-tex-engine-is-used
\usepackage{iftex}
\ifpdftex
	\typeout{^^J *** PDF MODE ***}
	\usepackage{cmap} % Make PDF files searchable and copyable
	\usepackage[utf8]{inputenc}
	\usepackage{scrwfile}
\fi
\ifluatex
	\typeout{^^J *** LuaLaTeX MODE ***}
\fi


% FONTES USADAS!!!
\usepackage{lmodern}
\usepackage{marvosym}
% bbding tem cross também
\let\Cross\relax
\usepackage{bbding}
% Preciso de emoji
\usepackage{fontspec}
\newfontfamily\DejaSans{DejaVu Sans}


%%%%%%%%%%%%%%%%%%%% l3packages
% This collection contains implementations for aspects of the LATEX3 kernel, dealing with higher-level ideas such as the Designer Interface
% pacotes i3packages
% frações mais flexíveis
\usepackage{xfrac}
% provides a high-level interface for declaring document commands
\usepackage{xparse}
%%%%%%%%%%%%%%%%%%% FIM l3packages


% to use \currenttime
\usepackage{datetime}

% avoid pdf warning messages from pdflatex
%\pdfminorversion=6

%\usepackage{kpfonts}
%\usepackage{showframe}
%\usepackage[a4paper]{geometry}
%Letra  de início de parágrafo
%\usepackage{lettrine}

%
% F I G U R A S
%

% controle melhor dos captions
% Captions e SUB FIGURAS (subcations resolve esse problema melhor)
\usepackage[centerlast,font={small},figurewithin=chapter, tablewithin=chapter]{caption}
\setlength{\captionmargin}{1cm}
\usepackage{subcaption}
% usa o H em imagens
%\usepackage{float}

% pacote para gerenciar quotes pequenos e grandes
%tem o comando \enquote
\usepackage[style=brazilian]{csquotes}
% from the manual
\renewcommand*{\mkcitation}[1]{ #1}
% magic for better quotes
%https://latex.org/forum/viewtopic.php?t=5444
\newenvironment*{smallquote}
   {\quote\footnotesize}
   {\endquote}
\SetBlockEnvironment{smallquote}


% vamos usar o biblatex, recomendação
%citestyle=alphabetic,bibstyle=authortitle
\usepackage[ citestyle=authoryear,articlein=false,
style=ext-authoryear-comp
,natbib=true,backref=true]{biblatex}

% fix "acedido" por algo mais razoável
% fiz "e alli" por "et al."
%\DefineBibliographyStrings{portuguese}{%
%  urlseen={Disponível em}
  %,andothers={et al.},andmore={et al.}
%  }
 % USE and other no campo author para forçar et al.
%\bibliographystyle{plainnat} %without url for book entries
% https://tex.stackexchange.com/questions/445858/changing-reference-style-in-biblatex
%https://tex.stackexchange.com/questions/10682/suppress-in-biblatex
% http://linorg.usp.br/CTAN/macros/latex/contrib/biblatex-contrib/biblatex-ext/biblatex-ext.pdf


%add document elements like a bibliography or an index to the Table of Contents
%\usepackage[nottoc,notlof,notlot]{tocbibind}


% novas keys de trim e valign (usado em vários
% tabelas com imagens
% export permite usar no includegraphics (exporta para ele)
\usepackage[export]{adjustbox}
%\graphicspath{ {./Images/} }





%\usetikzlibrary{3d}



% esse parametro evita
% que o texto seja separado da imagem
% e facilita (muito) tratar o tamanho
\usepackage[inkscapelatex=false]{svg}

%controla onde ficam os floats
% não queremos que pulem uma entrada
% usa o commando \FloatBarrier
\usepackage[section]{placeins}

% para addlinespace e toprule
\usepackage{array}
\usepackage{booktabs}
\usepackage{multirow}
\usepackage{multicol}

% permite novos tipos de colunas em Tabelas
% e ainda uma definição dinâmica
\usepackage{tabularx}
\newcolumntype{T}{>{\centering \arraybackslash}X}

% icons do CC, tem comandos conjuntos
\usepackage{ccicons}
% http://linorg.usp.br/CTAN/fonts/ccicons/ccicons.pdf

%%%%%%%%%%%% ENUMITEM configurado
\usepackage{enumitem}
% queremos menos espaço entre os itens de uma lista
\setlist{nosep}
% criando uma check-box list
\newlist{todolist}{itemize}{2}
\setlist[todolist]{label=$\square$}
% https://tex.stackexchange.com/questions/13463/specifying-bullet-type-when-using-itemize#
% o normal é \circle - e * (muito feio)
%https://texblog.org/2008/10/16/lists-enumerate-itemize-description-and-how-to-change-them/
\renewcommand{\labelitemi}{$\bullet$}
\renewcommand{\labelitemii}{$\circ$}
\renewcommand{\labelitemiii}{$\diamond$}
\renewcommand{\labelitemiv}{$\circ$}
% colocando . entre 1a para ficar 1.a
% nos \ref para \labels
% https://tex.stackexchange.com/questions/288407/no-dots-in-the-cross-reference-to-an-item-from-enumerate/288412#288412
% I want dots
\makeatletter
\renewcommand\p@enumii{\theenumi.}
\renewcommand\p@enumiii{\theenumi.\theenumi.}
\makeatother
%%%%%%%%%%%% ENUMITEM END

\usepackage{fancyvrb}


% Controlar a Marca Dagua
%\usepackage{draftwatermark}
%\SetWatermarkText{DRAFT}
%\SetWatermarkScale{5}
%\SetWatermarkColor[gray]{0.90}
%
% A M S -  PACKAGES
%
\usepackage{amsmath}
\usepackage{amssymb}
\usepackage{amsfonts}
\usepackage{amsthm}
\usepackage{mathtools}%

% para os exercícios
%\usepackage{exsol}
%\renewcommand{\exercisename}{Exercício}
%\renewcommand{\exercisesname}{Exercícios}
%\renewcommand{\solutionname}{Solução}
%\renewcommand{\solutionsname}{Soluções}
%\renewcommand{\seriesname}{Série}



% minicontent
%https://tex.stackexchange.com/questions/430594/use-minitoc-with-koma-script-scrbook
\usepackage{etoc}
\newcommand{\chaptertoc}[1][Conteúdo]{
	\etocsettocstyle{\addsec*{#1\\\rule{\textwidth}{0.4pt}}}
	{\bigskip}
	\etocsettocdepth{1}
	\localtableofcontents
}


% para caixas legais
\usepackage{tcolorbox}


% Títulos mais legais ver nos comandos
\usepackage[Bjornstrup]{fncychap}
\ChTitleVar{\raggedleft\Huge\sffamily\bfseries%
%\color{blue}
}
%\usepackage{titlesec}
%\titleformat{\chapter}[hang]{\Huge\bfseries}{\thechapter\hsp\textcolor{gray75}{|}\hsp}{0pt}{\Huge\bfseries}

%Para usar casos de uso, verificar no capítulo
%comandos copiados da rede
\usepackage{Comandos/usecases}

% Usando listagens

%não queremos figuras depois das footnotes
\usepackage[bottom]{footmisc}

\usepackage{etoolbox} % SEM USO AINDA

\usepackage{indentfirst}
\usepackage{wrapfig}


\usepackage{verse}
\newcommand{\attrib}[1]{%
    \nopagebreak{\raggedleft\footnotesize #1\par}}
\renewcommand{\poemtitlefont}{\normalfont\large\itshape\centering}

% Makeindex
% Não está gerando os índices
\usepackage{imakeidx}
\makeindex

% para os novos floats, como os gxinfobox
\usepackage{newfloat}

% para poder usar url
\usepackage[hidelinks]{hyperref}
% Glossários
%\usepackage[acronym,nomain]{glossaries}
% Automake is not working welll, use makeglossaries Livro...
%\usepackage[acronym,automake,docdef=restricted]{glossaries-extra}
%\makeglossaries
%\input{Extras/acronimosrt.tex}

\usepackage{lmodern}% scalable font to actually get the large numbers
